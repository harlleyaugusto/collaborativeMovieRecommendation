\documentclass[brazil,a4paper,11pt]{article}

\usepackage[num]{abntex2cite}
\usepackage[utf8]{inputenc}
\usepackage{sbc-template}

%\usepackage{algpseudocode}
\usepackage[T1]{fontenc}
\usepackage{amsfonts}
\usepackage{graphicx}
\usepackage[brazil]{babel}
\usepackage{lmodern}
\usepackage{amsmath}
\usepackage{float}
\usepackage{sbc-template}
\usepackage{setspace}
%\usepackage{algorithm}
\usepackage{epstopdf}
%\usepackage{algorithmic}
\usepackage{subfigure}

\onehalfspacing
\sloppy

\title{Trabalho prático 1: Collaborative Movie Recommendation}

\author{Harlley Augusto de Lima}
\address{Universidade Federal de Minas Gerais\\
Instituto de Ciências Exatas\\
\email{harlley@dcc.ufmg.br}}


\begin{document}
\maketitle

\section{Introdução}

Para esse trabalho prático é implementado um sistema de recomendação baseado em filtragem colaborativa (FC). Para tanto, é implementado a filtragem colaborativa baseada em na similaridades de usuários e na similaridades de itens. 

Na filtragem colaborativa baseada em usuários, a avaliação dada por usuários similares ao usuário alvo servem como base para serem feitas para o usuário alvo. Com o objetivo de determinar os usuários similares ao usuários alvo, sua similaridade com os demais usuários é computada. Em seguida, essa similaridade é utilizada para ponderar a predição final. Por outro lado, a filtragem colaborativo baseada em item para fazer as recomendações para o item alvo, o primeiro passo é determinas o conjunto de itens que são similares ao item alvo. Da mesma forma que FC baseada em usuário, é necessário implementar uma métrica de similaridade utilizada para identificar as similaridade entre os itens. Em seguida, esse valor é utilizado para ponderar a predição final para o item alvo.

Esse trabalho tem foi implementado um sistema de recomendação colaborativa de filmes. Para tal, foi implementado o FC baseada em item e baseada em usuário. Sendo que a implementação baseada em item alcançou melhores resultados no sistema Kaggle\footnote{https://www.kaggle.com/}. A seguir, na Seção Implementação são apresentados os detalhes de implementação juntamente com a análise de complexidade e na seção Resultados são apresentados os resultados da implementação.

\section{Implementação}

Essa seção mostra os detalhes de implementação dos componentes da FC baseada em item e em usuários. Para a implementação do sistema de recomendação foram feitas duas implementações. Como a matriz de utilizada é esparsa, a primeira implementação feita era baseada na estrutura \texttt{map}. Ou seja, a matriz de utilidade era implementada em \texttt{map}. Entretanto, tal implementação se mostrou lenta e excedia o tempo total de cinco minutos imposto na especificação. Dessa forma, essa implementação não foi considera, mas pode ser acessada no repositório do presente trabalho\footnote{Branch master com a implementação baseada em \texttt{map}:\\ https://github.com/harlleyaugusto/collaborativeMovieRecommendation}.

Assim, a implementação considerada no trabalho é baseada em matriz. Assim, é a matriz de utilizada é implementada como uma matriz de bidimensional. Os detalhes da implementação serão mostrados a seguir.


\subsection{FC baseada em itens}

\paragraph{Cosseno} 

\paragraph{Matriz de utilidade}

\paragraph{Map de itens}

\paragraph{Matriz de similaridade}


\subsection{FC baseada em usuários}

\paragraph{Cosseno} 

\paragraph{Matriz de utilidade}

\paragraph{Map de itens}

\section{Dificuldades}

\section{Resultados}


\small
\begin{thebibliography}{}
\bibitem{1}Thomas H. Cormen, Charles E. Leiserson, Ronald L. Rivest, Cliford Stein. Algoritmos Teoria e Pratica. Segunda Edição.
\bibitem{2}Nivio Ziviani. Projeto de Algoritmos. Segunda Edição.
\end{thebibliography}

%\bibliographystyle{abbrv}
%\bibliography{simple}

\end{document}