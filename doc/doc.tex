\documentclass[brazil,a4paper,11pt]{article}

\usepackage[num]{abntex2cite}
\usepackage[utf8]{inputenc}
\usepackage{sbc-template}

%\usepackage{algpseudocode}
\usepackage[T1]{fontenc}
\usepackage{amsfonts}
\usepackage{graphicx}
\usepackage[brazil]{babel}
\usepackage{lmodern}
\usepackage{amsmath}
\usepackage{float}
\usepackage{sbc-template}
\usepackage{setspace}
%\usepackage{algorithm}
\usepackage{epstopdf}
%\usepackage{algorithmic}
\usepackage{subfigure}
\usepackage{array}
\newcolumntype{P}[1]{>{\centering\arraybackslash}p{#1}}
\onehalfspacing
\sloppy

\title{Trabalho prático 1: Collaborative Movie Recommendation}

\author{Harlley Augusto de Lima}
\address{Universidade Federal de Minas Gerais\\
Instituto de Ciências Exatas\\
\email{harlley@dcc.ufmg.br}}


\begin{document}
\maketitle

\section{1. Introdução}

Nesse trabalho prático é proposto um sistema de recomendação baseado em filtragem colaborativa (FC) aplicado no contexto de filmes. Para tanto, é implementada a filtragem colaborativa baseada em similaridades entre usuários e na similaridades entre itens. 

Na filtragem colaborativa baseada em usuários, os \textit{ratings} dados pelos usuários similares ao usuário alvo servem como base para realizar as predições. Com o objetivo de determinar os usuários similares ao usuário alvo, a similaridade entre esses deve ser computada. Em seguida, essa similaridade é utilizada para ponderar a agregação dos \textit{scores}, gerando a predição final. Por outro lado, também foi implementada a filtragem colaborativa baseada em item. Nessa abordagem, o objetivo é fazer recomendações para o item alvo e o primeiro passo é determinar o conjunto de itens que são similares ao item alvo. Da mesma forma que FC baseada em usuário, é necessário implementar uma métrica de similaridade a ser utilizada para identificar a similaridade entre os itens. Em seguida, esse valor de similaridade é utilizado para ponderar a predição final do item alvo.

Nesse trabalho foi implementado um sistema de recomendação colaborativa de filmes. Para tal, foi implementada a FC baseada em item e baseada em usuário, sendo que a implementação baseada em item alcançou melhores resultados no sistema Kaggle\footnote{https://www.kaggle.com/}. A seguir, na Seção 2 são apresentados os detalhes de implementação juntamente com a análise de complexidade, na seção Seção 3 a avaliação experimental do método. Por fim, na Seção 4 é apresentada a conclusão e as dificuldades encontradas no trabalho.

\section{2. Implementação}

Essa seção mostra os detalhes de implementação dos componentes da FC baseada em item e em usuários. Para a implementação do sistema de recomendação foram feitas duas implementações. Como a matriz de utilidade é esparsa, a primeira implementação feita foi baseada na estrutura \texttt{hash}. Ou seja, inicialmente a matriz de utilidade era implementada em \texttt{hash}. Entretanto, tal implementação se mostrou lenta e excedia o tempo total de cinco minutos imposto na especificação. Dessa forma, essa implementação não foi considerada, mas pode ser acessada no repositório do presente trabalho\footnote{Branch master com a implementação baseada em \texttt{map}:\\ https://github.com/harlleyaugusto/collaborativeMovieRecommendation}.

Assim, a matriz de utilidade é implementada como uma matriz bidimensional e os demais detalhes da implementação serão mostrados a seguir.


\subsection{2.1 FC baseada em itens}

Para o desenvolvimento dessa abordagem os seguintes componentes devem ser implementados: métrica de similaridade, agregação da avaliação, normalização dos dados e outros componentes adicionais. Dessa forma, as estruturas utilizadas e a análise de complexidade de cada componente são detalhados a seguir. Para a análise de complexidade, considere que existam $m$ usuários e $n$ itens.

\paragraph{Métrica de similaridade} Conforme mencionado anteriormente, nessa abordagem é necessário computar a similaridade entre os itens. Como apresentado em~\cite{Jannach2010}, a métrica de similaridade que apresenta melhores resultados é o cosseno. O maior problema com essa abordagem é que usuários podem prover \textit{ratings} com escalas diferentes. Por exemplo, um usuário pode avaliar a maioria dos itens com valores altos, enquanto outros podem avaliar a maioria de forma negativa. Portanto, nesse trabalho é utilizado a métrica \textit{adjusted cosine}, em que é subtraída a média de \textit{score} do usuário de seus \textit{ratings}. Assim, a similaridade entre o item $a$ e $b$ pode ser calculada conforme definida na Equação~\ref{cosine}.

\begin{equation}
\label{cosine}
 sim(a,b) = \frac{\sum_{p \in P}(r_{a,p} - \bar{r_a})(r_{b,p} - \bar{r_b})}           {\sqrt{\sum_{p \in P}(r_{a,p} - \bar{r_a})^2} \sqrt{\sum_{p \in P}(r_{b,p} - \bar{r_b})^2}}
\end{equation}

\noindent A ordem de complexidade para calcular a similaridade dos itens é $O(m)$. 

\paragraph{Hash de itens} Para o cálculo do \textit{adjusted cosine}, é necessário saber quais usuários avaliaram os itens para os quais se deseja calcular a similaridade. Do contrário, o tempo para calcular a similaridade de dois itens seria muito alto, pois seria necessário multiplicar todas as linhas das duas colunas que representam os itens. Assim, é criado um \texttt{hash} de usuários para armazenar a lista de usuários que avaliaram um determinado item. Dessa forma, apenas serão multiplicados na computação da similaridade os \textit{ratings} dados pelos usuários que avaliaram os itens que estão sendo comparados. A estrutura de \texttt{hash} foi utilizada por prover um acesso rápido à lista de usuários que avalariam o item. 

\paragraph{Escolha da vizinhança} Para a escolha dos itens a serem utilizados para o cálculo da predição, foram escolhidos os $k$ itens mais similares de acordo com o \textit{adjusted cosine} ao item que se desejava fazer a predição para o usuário alvo. Dessa forma, afim de obter resultados mais satisfatórios torna-se necessária a variação da quantidade $k$ de itens. A complexidade da escolha da vizinhança é de $O(n)$ para ordenar o vetor de similaridade.

\paragraph{Matriz de utilidade} Como mostrado na Equação~\ref{predItem}, a todo momento é necessário acessar o \textit{rating} dado por um usuário em um determinado item. Para facilitar tal acesso, a matriz de utilidade é implementada com uma matriz bidimensional. Além disso, para computar a similaridade é necessário subtrair a média dos \textit{score} do usuário com o valor de \textit{score} que ele deu para o item. Dessa forma, a matriz de utilidade é criada com tal subtração já realizada, e não com o valores reais de \textit{score}. 

Por fim, foi necessário mapear os usuários e os itens para linhas e colunas da matriz. Para tal, foi criado um identificador que varia entre 0 e a quantidade total de usuários na base, sendo mapeados nas linhas da matriz. De forma similar, foi criado um identificador de 0 a quantidade total de itens na base para mapear os itens nas colunas. A estrutura de matriz provê um acesso rápido aos valores de \textit{ratings}, visto que para acessar tais valores basta o identificadores do item e do usuário.

\paragraph{Agregação das avaliações}  Para agregar as avaliações de cada item, foi utilizada a média ponderada para o cálculo da predição final. Assim, a ponderação da nota de cada item é feita utilizando a similaridade do item avaliado posteriormente pelo usuário alvo com o item que se deseja fazer a predição. Assim, a predição de um item $p$ para um usuário $a$ é dada pela Equação~\ref{predItem}.

\begin{equation}
\label{predItem}
pred(a,b) = \bar{r_a} + \frac{\sum_{b\in N} sim(a,b) * (r_{b,p} - \bar{r_b})}{\sum_{b \in N} |sim(a,b)|}
\end{equation}

A complexidade final para a predição final é $O(m^2n)$, que corresponde a calcular a similaridade de todos os itens e agregar as avaliações.


\paragraph{Matriz de similaridade} Para evitar que a similaridades de itens fossem computadas repetidamente, foi criada uma matriz para armazenar as similaridades já computadas. Essa matriz é quadrática e tem como dimensões a quantidade de itens total. Assim, para obter a similaridade de dois itens, é verificado se a similaridade de tais itens já não foi calculada e armazenada na matriz de similaridade. Caso não tenha sido calculada, o cálculo é feito e posteriormente armazenada na matriz de similaridades. Especificamente, essa matriz reduziu o tempo de execução do sistema, pois evitava que a similaridade de dois itens fosse calculada repetidamente. Além disso, o acesso a matriz era rápido, visto que necessitava apenas dos identificadores de cada item. 

Como a FC baseada em itens é uma abordagem de ordem quadrática, a criação dessa matriz diminui o tempo de processamento da predição final.


\subsection{2.2 FC baseada em usuário}

A abordagem de FC baseada em usuário é similar à abordagem baseada em item. A diferença direta é que a similaridade é calculada entre o usuário alvo e os demais usuários que avaliaram o item que se deseja calcular a predição. Sendo assim, as mesmas estruturas apresentadas anteriormente foram utilizadas nessa abordagem, com a exceção que o \textit{adjusted cosine} é calculado entre usuários. Além disso, foi necessário criar um \textit{hash} para armazenar a lista de itens avaliados por cada usuário. Tal estrutura agiliza a computação de similaridade, pois não é necessária a multiplicação de todas as colunas dos usuários que estão sendo comparados. Ou seja, a computação da similaridade considera apenas os itens avaliados em comum pelos dois usuários. 

Visto que essa abordagem é semelhante à baseada em item, a complexidade é também semelhante, sendo um algoritmo de ordem quadrática.

\section{3. Resultados}

Nessa seção são apresentados os resultados referentes a abordagem baseada em item, visto que tal implementação obteve melhor resultado no Kaggle.

Na implementação do sistema, foi proposta a utilização da matriz de similaridade para evitar a computação de similaridades de itens já calculados anteriormente. A Tabela~\ref{comXsem} mostra o tempo de execução do algoritmo com a utilização da matriz de similaridade e sem a matriz de similaridade, considerando toda a vizinhança. Como pode ser visto, a diferença entre o tempo de execução é cerca de 90 segundos a menos utilizando a matriz de similaridade, trazendo uma melhora significativa.

\begin{table}[!t]
  \centering
  \begin{tabular}{P{6cm} | P{6cm} }
  \hline\hline
  Com a matriz de similaridade & Sem a matriz de similaridade \\
       1m25.090s          &   2m54.018   \\ \hline\hline
  \end{tabular}
    \caption{Tempo de execução com e sem a matriz de similaridade, considerando toda a vizinhança.}
      \label{comXsem}
\end{table}

A Tabela~\ref{variacaoK} mostra o tempo de execução do sistema a medida que o número $k$ de itens semelhantes aumenta, tanto com a matriz de similaridade quanto sem tal matriz. Comparando as duas colunas fica claro que a abordagem que utiliza a matriz de similaridade possui sempre um tempo de processamento menor. Além disso, a medida que $k$ aumenta, o tempo de execução do sistema se mantém praticamente constante, utilizando a matriz de similaridade. Tal fato também ocorre sem a utilização da matriz, ainda que com $k$ igual a 120 o tempo de execução aumenta quase 16 segundos em relação a $k$ igual a 1. 

Em relação a qualidade das predições obtido no Kaggle, o melhor resultado obtido foi com a abordagem baseada em item, que alcançou RMSE igual a 2.30395. Ainda tentando melhor os resultados, foram feitas variações no tamanho $k$ da vizinhança, não acarretando em melhora significativa. Diante de tal resultado não satisfatório, foi implementada a FC baseada em usuário, que também não melhorou a qualidade das predições. 

\begin{table}[!t]
  \centering

  \begin{tabular}{P{2cm} | P{4cm} | P{4cm} }
  \hline\hline
  $k$ & Com a matriz de similaridade (s) & Sem a matriz de similaridade (s) \\ \hline \hline
      1 & 74.006           & 168.470  \\
      5 &  74.196        &  168.179 \\
      10 &   74.857          &  175.885 \\
      15 &    74.861         &  170.551 \\
      30 &     74.386       &   164.456\\
      60 &      73.206       & 171.873  \\
      90 &      78.649      &  169.900 \\
      120 &       83.895   &  184.972 \\ \hline\hline
  \end{tabular}
    \caption{Tempo de execução com e variando o tamanho $k$ da vizinhança.}
  \label{variacaoK}
\end{table}

\section{4. Conclusão e dificuldades}

Nesse trabalho foi desenvolvido um sistema de recomendação de filmes. Para tal, foram desenvolvidas duas abordagens: filtragem colaborativa baseada em item e baseada em usuário. Sendo que a abordagem baseada em item obteve melhores resultados de acordo com os resultados apresentados no Kaggle. Para implementação de tal sistema as seguintes dificuldades foram encontradas:

\begin{enumerate}
 \item \textbf{Implementação utilizando C++.} A falta de conhecimento pleno na linguagem tornou o tempo de implementação ainda maior. Em particular, a passagem de parâmetro padrão nessa linguagem é por cópia. Isso deixava a execução do sistema desenvolvido bastante lento. Entretanto, passando os parâmetros por referência diminuiu consideravelmente o tempo de execução.  Esse ponto atrasou consideravelmente o desenvolvimento do trabalho.
 
 \item \textbf{Otimização do código.} Na primeira versão do sistema, a matriz de utilidade foi implementada utilizando \texttt{hash}. Entretanto, tal implementação demorava cerca de 8 minutos para gerar todas as predições do arquivo de entrada. Assim, foi implementada uma nova versão em que a matriz de utilidade é implementada com uma matriz bidimensional, reduzindo o tempo de processamento. Isso também atrasou a implementação do sistema.
 
 \item \textbf{Otimização dos resultados.} A primeira abordagem implementada foi a FC baseada em item. Como tal, abordagem não apresentou bons resultados no Kaggle, foi também implementada a FC baseada em usuário, que também não obteve bons resultados. Diante disso, foram feitos vários esforços para melhorar a qualidade das predições que não surtiram efeitos. 
\end{enumerate}

Apesar dos resultados das predições não serem satisfatórios, com esse trabalho foi possível passar por todas as etapas de desenvolvimento da abordagem colaborativa baseada em item e em usuário, sendo elas: seleção da vizinhança, agregação dos \textit{ratings} e normalização dos dados. Além disso, foi feito um esforço considerável para manter o tempo de execução baixo e, conforme apresentado, o tempo de execução considerando toda a vizinhança na abordagem de FC baseada em item é de cerca 1m25s.


%\bibliographystyle{ieeetr}
\bibliography{simple}

\end{document}