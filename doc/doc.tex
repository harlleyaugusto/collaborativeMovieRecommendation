\documentclass[brazil,a4paper,11pt]{article}

\usepackage[num]{abntex2cite}
\usepackage[utf8]{inputenc}
\usepackage{sbc-template}

%\usepackage{algpseudocode}
\usepackage[T1]{fontenc}
\usepackage{amsfonts}
\usepackage{graphicx}
\usepackage[brazil]{babel}
\usepackage{lmodern}
\usepackage{amsmath}
\usepackage{float}
\usepackage{sbc-template}
\usepackage{setspace}
%\usepackage{algorithm}
\usepackage{epstopdf}
%\usepackage{algorithmic}
\usepackage{subfigure}

\onehalfspacing
\sloppy

\title{Trabalho prático 1: Collaborative Movie Recommendation}

\author{Harlley Augusto de Lima}
\address{Universidade Federal de Minas Gerais\\
Instituto de Ciências Exatas\\
\email{harlley@dcc.ufmg.br}}


\begin{document}
\maketitle

\section{1. Introdução}

Para esse trabalho prático é implementado um sistema de recomendação baseado em filtragem colaborativa (FC). Para tanto, é implementado a filtragem colaborativa baseada em na similaridades de usuários e na similaridades de itens. 

Na filtragem colaborativa baseada em usuários, a avaliação dada por usuários similares ao usuário alvo servem como base para serem feitas para o usuário alvo. Com o objetivo de determinar os usuários similares ao usuários alvo, sua similaridade com os demais usuários é computada. Em seguida, essa similaridade é utilizada para ponderar a predição final. Por outro lado, a filtragem colaborativo baseada em item para fazer as recomendações para o item alvo, o primeiro passo é determinas o conjunto de itens que são similares ao item alvo. Da mesma forma que FC baseada em usuário, é necessário implementar uma métrica de similaridade utilizada para identificar as similaridade entre os itens. Em seguida, esse valor é utilizado para ponderar a predição final para o item alvo.

Esse trabalho tem foi implementado um sistema de recomendação colaborativa de filmes. Para tal, foi implementado o FC baseada em item e baseada em usuário. Sendo que a implementação baseada em item alcançou melhores resultados no sistema Kaggle\footnote{https://www.kaggle.com/}. A seguir, na Seção Implementação são apresentados os detalhes de implementação juntamente com a análise de complexidade e na seção Resultados são apresentados os resultados da implementação.

\section{2. Implementação}

Essa seção mostra os detalhes de implementação dos componentes da FC baseada em item e em usuários. Para a implementação do sistema de recomendação foram feitas duas implementações. Como a matriz de utilizada é esparsa, a primeira implementação feita era baseada na estrutura \texttt{map}. Ou seja, a matriz de utilidade era implementada em \texttt{map}. Entretanto, tal implementação se mostrou lenta e excedia o tempo total de cinco minutos imposto na especificação. Dessa forma, essa implementação não foi considera, mas pode ser acessada no repositório do presente trabalho\footnote{Branch master com a implementação baseada em \texttt{map}:\\ https://github.com/harlleyaugusto/collaborativeMovieRecommendation}.

Assim, é a matriz de utilizada é implementada como uma matriz de bidimensional. Os detalhes da implementação serão mostrados a seguir.


\subsection{2.1 FC baseada em itens}

Para implementação dessa abordagem os seguintes componentes devem ser implementados: métrica de similaridade, agregação da avaliação e normalização dos dados. Dessa forma, cada componente e estruturas utilizadas nessa abordagem são detalhados a seguir. 

\paragraph{Métrica de similaridade} Conforme apresentado anteriormente, nessa abordagem é necessário computar a similaridade entre os itens. Como apresentado em~\cite{Jannach2010}, a métrica de similaridade que apresenta melhores resultados é o cosseno. O maior problema com essa abordagem é que usuários podem prover \textit{ratings} com escalas diferentes. Por exemplo, um usuário pode avaliar a maioria dos itens com valores mais altos, enquanto outros podem avaliara a maioria de forma negativa. Para isso nesse trabalho é utilizado a métrica \textit{adjusted cosine}, em que é subtraída a média do usuários de seus \textit{ratings}. A similaridade entre o item $a$ e $b$ pode ser calculada conforme definida na Equação~\ref{cosine}.

\begin{equation}
\label{cosine}
 sim(a,b) = \frac{\sum_{p \in P}(r_{a,p} - \bar{r_a})(r_{b,p} - \bar{r_b})}           {\sqrt{\sum_{p \in P}(r_{a,p} - \bar{r_a})^2} \sqrt{\sum_{p \in P}(r_{b,p} - \bar{r_b})^2}}
\end{equation}


\paragraph{Agregação das avaliações}  Para agregar as avaliações de cada item, foi utilizada a média ponderada para o calculo da predição final. Assim, a ponderação da nota de cada item é feita utilizando a similaridade do item avaliado posteriormente pelo usuário alvo com o item que se deseja fazer a predição. Assim a predição de um item $p$ para um usuário $a$ é dada pela Equação~\ref{predItem}.

\begin{equation}
\label{predItem}
pred(a,b) = \bar{r_a} + \frac{\sum_{b\in N} sim(a,b) * (r_{b,p} - \bar{r_b})}{\sum_{b \in N} |sim(a,b)|}
\end{equation}

\paragraph{Escolha da vizinhança} Para a escolha dos itens a serem utilizados para o calculo da predição, foram escolhidos os $k$ itens mais similares ao item que se desejava fazer a predição para o usuário alvo. Dessa forma, afim de obter resultados mais satisfatórios torna-se necessária a variação de $k$.

\paragraph{Matriz de utilidade} Como mostrado na Equação~\ref{predItem}, a todo momento é necessário acessar o \textit{rating} dado por um usuário em um determinado item. Para facilitar tal acesso, a matriz de utilidade é implementada com uma matriz bidimensional. Além disso, como para computar a similaridade é necessário subtrair a media de score do usuário com o valor de score que ele deu para o item, a matriz de utilidade é criada com tal diferença, e não com o valores reais de score. 

Além disso, foi necessário mapear os usuários e os itens para linhas e colunas da matriz. Para tal, foi criado um identificador que variava entre 0 e quantidade total de usuários na base, para mapear os usuários. De forma similar, foi criado um identificador de 0 a quantidade total de itens na base, para mapear os itens.

\paragraph{Hash de itens} Para o calculo do \textit{adjusted cosine}, é necessário saber quais usuários avaliaram os itens que se deseja calcular a similaridade. Do contrário, o tempo para calcular a similaridade de dois itens seria muito alto, pois seria necessário a multiplicação de todas as linhas das duas colunas que representam os itens. Assim, é criado um \texttt{hash} de itens para armazenar a lista de usuários que o avaliaram. 

\paragraph{Matriz de similaridade} Para evitar que a similaridades de itens fossem computadas repetidamente, foi criada uma matriz para armazenar as similaridades computadas. Assim, para obter a similaridade de dois itens, é verificado se a similaridade de tais itens já não foi calculada e armazenada na matriz de similaridade. Caso não tenha sido calculada, o calculo é feito e armazenado na matriz.


\subsection{2.2 FC baseada em usuário}

A abordagem de filtragem colaborativa baseada em usuário é similar à abordagem baseada em item. A diferença direta é que a similaridade é calculada entre usuário o usuário alvo e os demais usuários que avaliaram o item que se deseja calcular a predição. Sendo assim, as mesmas estruturas apresentadas anteriormente foram utilizadas nessa abordagem, com a exceção que o \textit{adjusted cosine} é calculado entre usuários. 

Além disso, foi necessário criar ...

\paragraph{Map de usuários}

\section{3. Resultados}

\section{4. Dificuldades}




%\small
% \begin{thebibliography}{}
% \bibitem{1}Thomas H. Cormen, Charles E. Leiserson, Ronald L. Rivest, Cliford Stein. Algoritmos Teoria e Pratica. Segunda Edição.
% \bibitem{2}Nivio Ziviani. Projeto de Algoritmos. Segunda Edição.
% \end{thebibliography}

\bibliographystyle{ieeetr}
\bibliography{simple}

\end{document}